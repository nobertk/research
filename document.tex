\documentclass[]{report}


% Title Page
\title{STARTING A BLOG}
\author{BARUGAHARE NOBERT}


\begin{document}
\maketitle

\begin{abstract}

	A blog, by definition is a frequently updated online personal journal or diary. It is a place to express yourself to the world, a place to share your thoughts and your passions. Really, it’s anything you want it to be. Blog is a short form for the word weblog and the two words are used interchangeably. Originally blogs were known primarily as places for people to write about their day-to-day activities, today people write about far more interesting topics. Blogging has suddenly caught-on as a legitimate hobby, to meet this demand some amazing tools have been created that will allow anyone, even people with very little knowledge of computers, to have their own blog. If you can find your way onto the Internet and follow some basic instructions you can have your own blog.\\
	
\subsection(Why do people Blog)
	
	There are still many people who like to share the details of their days. On the other hand there are bloggers who give almost no detail about their lives, but write instead about a hobby or interest of theirs. They may dedicate their blog to something they are passionate about. In fact, today’s blogs can provide hair tips, up-to-date news, technical information, celebrity scandal, political rumor, gets people involved in volunteering, advice on investments as well as there being blogs about niche topics like cooking, health, gardening, sport, blogging blogs and of course many personal blogs and quite a few strange blogs.\\
	
\section{Getting started}
	
	Setting up a blog is really simple and straight forward these days. Before setting up one, one should know the following. Setting up a blog does not take long, however choosing a domain name and theme can take a while if you do not know what you want them to be. Creating quality content potential readers want to read can take even more time. One can create a blog for free if they want, but may have to pay for a few things like a domain name and hosting if you want to start a serious blog. You can set up a blog for fun, a hobby or even to make money. When all this is put into consideration, there are basically few steps needed to start a successful blog.
	
\section{Steps to create a blog}
	•	Pick a blog platform
	•	Choose a domain
	•	Get web hosting 
	•	Install the software and set up blog
	•	Select blog design and layout
	\\
	\\
	1.	Pick a blog platform
	When it comes to starting your blog you have the following options: free, freemium and self-hosted (recommended) platforms an example of the free platforms include WordPress, Blogger, Tumblr and Blog Engine. The benefit is that it’s free, but for a serious blogger having a blog name in the form –yourname.tumblr.com is a sign of inexperience and may not be taken seriously. \\
	Freemium means you will be given a trail period before you start to pay, one of the popular platforms is TypePad.\\
	Self-hosted platforms allow you to run a blog on your own domain, most popular is WordPress. Aside from following your domain registrar and web hosting company's rules, you’re fully in charge of your blog and its contents. To get stated on a self-hosting blog, you will need a domain name and web hosting provider like Bluehost.
	\\
	\\
	2.	Pick a domain name
	Your domain name will be the name by which you will be known online, no matter what you choose. It’s the unique address of your blog on the Internet. Your domain will be yours as long as you continue paying the annual fee. Users who know your domain can simply type it into their browser's address bar. Others will be able to discover your blog through search engines, so you definitely need to find a unique one.
	 \\
	 \\
	3.	Get web hosting
	Selecting reliable hosting services will be one of the most important decisions you make. To a great extent, the functionality and performance of your blog will depend on your hosting provider. The host makes sure your blog is available 24/7 to potential readers and it’s where your files are stored online.
	\\
	\\
	
	4.	Installing blog software and set up a blog
	There are many to pick from, one can use price and available functionality to help pick from the available.
	\\
	\\
	5.	Selecting a design and layout
	When installing is done the first thing your blog will need is a face, which is the design and layout. Choose a theme that looks great, but also works for your unique content. Readers will first notice the overall appearance of the blog, before even taking a look at the content.
	This will help anyone setup a blog fast without any large hindrance and probably will run it successfully. 
	
	
\end{abstract}

\end{document}          
